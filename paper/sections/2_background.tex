%!TEX root = ../paper.tex

\section{Text Classification in Marketing}
\label{sec:background}

This section shows related work from text classification and recommender systems.
Then we give a formal problem definition and show challenges, that have to be solved.

\subsection{Related Work}
The text classification task has been researched already in many other works.
Yang et~al.~\cite{yang1999re} have compared the performance of different text classifiers in their paper.
They tested linear classifiers, rule-based classifiers, k-nearest-neighbours and Na\"ive Bayes classifiers.
In their evaluation, the different classifiers did not differ by more than five percent.
McCallum et~al.~\cite{mccallum1998comparison} specifically focus on Na\"ive Bayes classifiers.
They found out, that Na\"ive Bayes classifiers perform good on small corpora and vocabulary sizes.

In the area of automatic product recommendations, research has focused on displaying ads in search engines.
Mehta et~al.~\cite{mehta2007adwords} present an algorithm to find ads to display given a user's search terms.
However, for search engines the input mostly is only a few words, while we use entire posts.
The concept is also different, because companies directly place bids on certain keywords.
This leads to a large preselection already.

There is also a wide research on recommender systems. 
Especially the content-based recommender systems~\cite{lops2011content} are similar to our problem.
Content-based recommender systems use the content to identify items ``similar to those a given user has liked in the past''~\cite{lops2011content}.
We also recommend posts to salesmen.
However, we do not have a history of posts.
We use marketing materials as similarity measure.

Furthermore, we focus not on recommending products to users automatically but on identifying posts for salesmen.
This is why we need to be sure that a recommended post expresses a need for a specific product.
Therefore, we think, that this paper represents a good beginning for the research in this field of text classification.


\subsection{Problem Definition}
\label{sec:background-problem}

In this section we formally define the \emph{noise to opportunity problem}.
We also talk about problems that occur while solving the task.

Let \acme be an example company that offers a wide range of different products.
As in every company, there are salesmen who want to get in contact with potential buyers of these products.
For different products, there are different salesman responsible and different potential buyers.
To find these buyers we help \acme to identify people on business networks, who seem to have a need for products offered by \acme.

We call our approach noise to opportunity conversion or the \nto problem.
We choose the word noise because of the amount of daily status updates in business networks, which makes it infeasible to listen to all of them.
Also, status updates are usually quite diverse: people write posts, which differ in length, use different words for the same thing, use  different abbreviations or even invent new words.
Nevertheless, there are many posts which are a good opportunity to get in contact with people who want to buy products a salesman's company offers.

Posts~\ref{post:demand-and-product}, ~\ref{post:demand-only}, and~\ref{post:product-only} show how posts in such business-oriented social networks look like.
We have to decide which of these posts are written by potential customers of \acme.

\begin{post}
	\centering
	\boxedTex{
		Hi, I'm looking for CRM advice.
		We're a gaming company, currently focused on a Slot Machine game on iOS and Android.
		We're in the process of finding a CRM platform to help us manage our player base.
		Do you have any recommendations?
	}
	\caption{The user wants to buy a new product, here a software for customer relationship management (CRM). Assuming that \acme sells this type of product, the system should make a recommendation.}
	\label{post:demand-and-product}
\end{post}

\begin{post}
	\centering
	\boxedTex{
		Hi, I am looking for your advice.
		Which car should I buy next?
		Do you have any recommendations
	}
	\caption{The user wants to buy something, but assuming that \acme does not sell cars, the system should not make a recommendation.}
	\label{post:demand-only}
\end{post}

\begin{post}
	\centering
	\boxedTex{
		Please, join our demo tomorrow. We give you an overview of different CRM systems.
		We also show how you can improve your customer relations to increase your sales.
	}
	\caption{The post is about a product offered by the company, but the user does not want to buy the product. The system should not make a recommendation.}
	\label{post:product-only}
\end{post}

We would only recommend Post~\ref{post:demand-and-product} to \acme.
Post~\ref{post:demand-and-product} expresses that the user is searching for a new customer relationship managment (CRM) system.
This product is offered by \acme.
In Post~\ref{post:demand-only}, the user is searching for a product.
However, this post is uninteresting for \acme because cars are not in \acme's product portfolio.
In contrast, Post~\ref{post:product-only} is about CRM systems, but the user is not interested in a new CRM system.
Therefore the post should not be shown to a salesman.

The decision, whether to recommend a post, depends on two characteristics:
First, the post should express that users needs something, e.g. a certain product.
Second, the product needed by the user should be offered by \acme.

Formally, the problem can be defined as:
Let $p \in POSTS$ be a post from a business-oriented social networking service.
Let $PRODUCTS$ denote the set of a company's products.
Then, the problem can be modelled as a function $\nto$, such that:
\begin{displaymath}
	\nto: POSTS \to PRODUCTS \cup \{NONE\}
\end{displaymath}
This function can also be seen as an assignment of a post to a salesman.
Usually, one salesman is not responsible for all products of \acme, but rather specializes in a few.
If the \nto function outputs $CRM$, then the post is shown to one of the CRM salesmen for review.

A classification of a post $p \in POSTS$ is $correct$, if $\nto(p)$ is indeed the product that seems to be needed by the user who posted $p$ or $NONE$ if the user does not need a product offered by the company \acme.
If $p$ is a post, then we propose $p$ to a salesman if $\nto(p) \neq NONE$.

We can optimize our algorithm for two use cases: helping salesmen find relevant posts or displaying advertisement automatically.
For helping salesmen it is important to have a high precision.
If we recommend a post to salesmen we want to be confident, that our recommendation is correct.
A high precision reduces the amount of unnecessary work.
For displaying advertisements automatically the recall should be high.
It is important to recommend products to as many users as possible.
It is acceptable if some of the recommendations are not correct.

In the following sections we explain how we approached the problem of approximating the~\nto function.
We achieve this by learning a classifier from a sample data set.
The classifier needs a data set with some classified documents.
Classified means, that they are already tagged with respect to whether our \nto function should recommend a product or not.
This is called a gold standard.
However, salesmen do not want to read through hundreds of posts to tag them manually.
Therefore, we need a different data source.
We decide to use advertisement brochures describing \acme's products.
As a consequence of using brochures a problem arises from the theoretical learning perspective.

From a probabilistic point of view the instances are drawn from an unknown distribution.
Learning means to find a function, which minimizes the expected loss with respect to this distribution~\cite{trafalis2000support}.
In our approach, training works on brochures, while later we want to classify social network posts.
They are not drawn from the same population which means that the distributions are different.
Therefore, the resulting function might not be the optimal with respect to classifying posts.
To overcome this, we take a closer look on the differences between brochures and posts:

 \begin{itemize}
 	\item
		\emph{Non-user perspective}:
		Brochures are written from the perspective of salesmen.
		However, we want to classify on user-written social network posts.
		As Post~\ref{post:product-only} illustrates, simply finding posts similar to the brochures is not sufficient, as we also need to consider whether the user actually wants feedback and recommendations.
	\item
		\emph{Document mismatch}:
		Brochures are written with a different intent as posts.
		There is a crucial difference in writing style and word choice between posts.
		In brochures there are words from the marketing domain.
		Also, brochures are much longer.
		Usually, they are at least one page long while social network posts consist only of a few sentences.
	\item
		\emph{Small corpus}:
		Using brochures in the learning phase is a problem, because there are only a few brochures for each product.
 \end{itemize}
