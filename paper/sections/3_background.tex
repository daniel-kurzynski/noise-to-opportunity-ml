%!TEX root = ../paper.tex

\section{Text Classification in Marketing}
\label{sec:background}

\subsection{Related Work}

\begin{itemize}
	\item Show related work for text classification (initial presentation)
\end{itemize}

\subsection{Problem Definition}
\label{sec:background-problem}

In this section we specify the problem we are going to solve. We also talk about problems that could occur while solving the task. 

\todo{change \acme definition}
Let \acme be our example software company that offers products like customer relationship management systems or human capital management systems. 
As in every company there are salesmen who want to get in contact with potential buyers of these products. 
To find these buyers we help \acme to identify people on business networks who seems to have a need for products offered by \acme.

\todo{Noise refers to both: amount and quality of input}
We call our approach Noise-To-Opportunity conversion or \nto problem.
We choose the word noise because of the amount of daily status updates in business networks which makes it infeasible to listen to all off them.
Nevertheless, there are many posts which are a good opportunity the get in contact with people who wants to buy products a salesman's company offers. 

Post~\ref{post:demand-and-product}, ~\ref{post:demand-only}, and~\ref{post:product-only} shows how posts in such business-oriented social networks could like. 
We have to decide which off these posts are written by potentially customers of \acme. 

\begin{post}
	\centering
	\boxedTex{
		Hi, I'm looking for CRM advice. 
		We're a gaming company, currently focused on a Slot Machine game on iOS and Android. 
		We're in the process of finding a CRM platform to help us manage our player base. 
		Do you have any recommendations?
	}
	\caption{The user wants to buy a new product, here a CRM system. The system should make a recommendation.}
	\label{post:demand-and-product}
\end{post}

\begin{post}
	\centering
	\boxedTex{
		Hi, I am looking for your advice. 
		Which car should I buy next?
		Do you have any recommendations
	}
	\caption{The user wants to buy something, but assuming that the company does not sell cars, the system should not make a recommendation.}
	\label{post:demand-only}
\end{post}

\begin{post}
	\centering
	\boxedTex{
		Please, join our demo tomorrow. We give you an overview of different CRM systems.
		We also show how you can improve your customer relations to increase your sales.
	}
	\caption{The post is about a product offered by the company, but it is no user wanting to buy the product. The system should not make a recommendation for it.}
	\label{post:product-only}
\end{post}

We would only recommend post~\ref{post:demand-and-product} to \acme.
Post~\ref{post:demand-and-product} expresses that the user is searching for a new CRM system.
This is offered by \acme.
In Post~\ref{post:demand-only} the user is searching for a product. 
However, this post is uninteresting for \acme because a car is not in \acme's product portfolio. 
In contrast post~\ref{post:product-only} is about CRM systems, but the user is not interested in a new CRM system.
Therefore the post should not be shown to a salesman.

The decision, whether to recommend a post, depends on two characteristics.
The post should express that users needs something, e.g. a product.
The product needed by the user should be offered by \acme.


Formally, the problem can be defined as: 
Each $p \in POSTS$ is a post from a business-oriented social networking service. $PRODUCTS$ denote the set of a company's product.
Then, the problem can be classified as a function $\nto$, such as:
\begin{displaymath}
	\nto: POSTS \to PRODUCTS \cup \{NONE\}
\end{displaymath}

\todo{salesmen are responsible for different products: only for one -> Use \nto function to detect posts that belong to salesman}

A classification of a post $p \in POSTS$ is $correct$, if $\nto(p)$ equates to the product that seems to be needed by the user who posted $p$ or equates to $NONE$ if the user does not need a product offered by the company. If $p$ is a post, then we propose $p$ to a salesman if $\nto(p) != NONE$.

However, we can optimize our algorithm for both use cases: Helping salesmen or displaying advertisement automatically.
For helping salesmen it is important to have a high precision. 
If we recommend a post to salesmen we should be definitely sure, that our recommendation is correct. 
A high precision reduces the unnecessary work.
For displaying advertisement automatically the recall should be high. 
It is important to recommend products to as many users as possible.
It is acceptable if some of theses recommendation are not correct.   

In the following chapters we explain how we build the~\nto function.
We archive this by learning a classifier from a sample data set. 
Therefore, we need a data set where for each document is already tagged whether our \nto function should recommend a product or not.
However, salesmen does not want to read through hundreds of posts to tag them manually.
We need a different data source.
We decide to use advertisement brochures describing \acme's products. 
Using brochures a problem arises from the theoretical learning perspective.

In a learning task the instances should be drawn from the same population in the training phase and during the classification.\nr
Here in the training phase we use brochures. Later on we want to classify posts. 
Both groups of documents are slightly different.
The following problems we are going to solve:

 \begin{itemize}
 	\item \emph{Non-user perspective}: Brochures are written from the perspective of salesmen. However, we want to classify on user-written social network posts.
 	As Post~\ref{post:product-only} illustrates simply finding posts similar to the brochures is not sufficient, as we also need to consider whether the user actually wants feedback and recommendations.
	\item \emph{Document mismatch}: Brochures are written with a different intend as posts. 
	Andere Worte kommen in Marketing vor, z. B. ``" . Also, they are more frequent as brochures are usually much longer (often at least one pdf page) while social network posts 
	There is a crucial difference in writing style and word choice between 
	\item \emph{Small corpus}: 
 \end{itemize}

% \begin{itemize}
% 	\item Who inspired your work?
% 	\item Which aspect do you improve?
% 	\item What are the basic foundations?
% \end{itemize}

%another problem:
%For each product the relative frequency of documents dealing with this product is different.
%The probability that a randomly chosen document belongs to a specific product should be the same in both groups.
%The classifier uses this probability in the training phase to improve the predictions.\nr