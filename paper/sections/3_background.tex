%!TEX root = ../paper.tex

\section{Text Classification in Marketing}
\label{sec:background}

\subsection{Related Work}

\begin{itemize}
	\item Show related work for text classification (initial presentation)
\end{itemize}

\subsection{Problem Definition}
\label{sec:background-problem}

In this section we specify the problem we are going to solve. We also talk about problems that could occur while solving the problem. 

Let \acme be our example software company that offers products like customer relationship management systems or human capital management systems. 
As in every company there are salesmen who want to get in contact with potential buyers of these products. 
To find these buyers we help \acme to identify people on business networks who seems to have a need for products offered by \acme.

We call our approach Noise-To-Opportunity conversion or \nto problem.
We choose the word noise because of the amount of daily status updates in business networks which makes it infeasible to listen to all off them.
Nevertheless, there are many posts which are a good opportunity the get in contact with people who wants to buy products a salesman's company offers. 

Post~\ref{post:demand-and-product}, ~\ref{post:demand-only}, and~\ref{post:product-only} shows how posts in such a business network could like. 
We have to decide which off these posts are written by potentially customers of \acme. 

\begin{post}
	\centering
	\boxedTex{
		Hi, I'm looking for CRM advice. 
		We're a gaming company, currently focused on a Slot Machine game on iOS and Android. 
		We're in the process of finding a CRM platform to help us manage our player base. 
		Do you have any recommendations?
	}
	\caption{The user wants to buy a new product, here a CRM system. The system should make a recommendation.}
	\label{post:demand-and-product}
\end{post}

\begin{post}
	\centering
	\boxedTex{
		Hi, I am looking for your advice. 
		Which car should I buy next?
		Do you have any recommendations
	}
	\caption{The user wants to buy something, but assuming that the company does not sell cars, the system should not make a recommendation.}
	\label{post:demand-only}
\end{post}

\begin{post}
	\centering
	\boxedTex{
		Please, join our demo tomorrow. We give you an overview of different crm systems.
		We also show how you can improve your customer relations to increase your sales.
	}
	\caption{The post is about a product offered by the company, but it is no user wanting to buy the product. The system should not make a recommendation for it.}
	\label{post:product-only}
\end{post}

We would only recommend post~\ref{post:demand-and-product} to \acme.
Post~\ref{post:demand-and-product} expresses that the user is searching for a new crm system.
This is offered by \acme.
In Post~\ref{post:demand-only} the user is searching for a product. 
However, this post is uninteresting for \acme because a car is not in \acme's product portfolio. 
In contrast post~\ref{post:product-only} is about crm systems, but the user is not interested in a new crm system.
Therefore the post should not be shown to a salesman.

The decision, whether to recommend a post, depends on two characteristics.
The post should express that users needs something, e.g. a product.
The product needed by the user should be offered by \acme.


Formally, the problem can be defined as: 
Each $p \in POSTS$ is a post from a business-oriented social networking service. $PRODUCT$ denote the set of a company's product.
Then, the problem can be classified as a function $\nto$, such as:
\begin{displaymath}
	nto: POSTS \to PRODUCT \cup \{NONE\}
\end{displaymath}

If $p$ is a post, then we propose $p$ to a salesman if $\nto(p) != NONE$.
A classification of a post $p \in POSTS$ is $correct$, if it is indeed the correct product recommended.

optimize for both use cases: For salesman and for machine.

In the following chapters we explain how we build this~\nto.
We archive this by learning from a sample dataset. 

small corpus

document-/length mismatch problem (theory vs practice)

listing to summarize


% \begin{itemize}
% 	\item Who inspired your work?
% 	\item Which aspect do you improve?
% 	\item What are the basic foundations?
% \end{itemize}
