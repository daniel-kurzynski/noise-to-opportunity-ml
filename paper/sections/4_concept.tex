%!TEX root = ../paper.tex
\section{Concept}
\label{sec:concept}

We now present our approach to the $nto$ problem.
The next section details our two-staged classification approach.
We then talk about what features we select and how we approach the \emph{\documentmismatch}.

\subsection{Two-staged classification}

In our example data set (see Section~\ref{sec:evaluation})

Only a few training data
Different characteristica
Example: Demand aber kein PRoduct
Schnittmenge der charakteristischen worte sehr klein
Schnittmenge der eigentlichen Dokumente auch sehr klein (only few positive examples)

only a few 

\begin{itemize}
	\item Two-staged classifier (why we did this)
	\item Feature selection
	\item Training data generation (using splits and random recombinations)
\end{itemize}

% \begin{itemize}
% 	\item Elaborate on your idea
% 	\item First give examples and than try to generalize
% \end{itemize}

\begin{figure}
	\centering
	\begin{subfigure}[t]{0.3\textwidth}
		\begin{tabular}{r | r}
			\textbf{Threads} & \textbf{Runtime}\\
			\hline
			1 & 29952\\
			2 & 15538\\
			4 & 10467\\
			8 & 7987
		\end{tabular}
		\caption{Computation of keyword distribution and identification of bursty intervals}
	\end{subfigure}~
	\begin{subfigure}[t]{0.3\textwidth}
		\begin{tabular}{r | r}
			\textbf{Threads} & \textbf{Runtime}\\
			\hline
			1 & 4649\\
			2 & 4430\\
			4 & 5429\\
			8 & 6427
		\end{tabular}
		\caption{Creation of events from keyword clustering}
	\end{subfigure}~
	\begin{subfigure}[t]{0.3\textwidth}
		\begin{tabular}{r | r}
			\textbf{Threads} & \textbf{Runtime}\\
			\hline
			1 & 514\\
			2 & 281\\
			4 & 250\\
			8 & 203
		\end{tabular}
		\caption{Matching of documents to events}
	\end{subfigure}
	\caption{Performance of each step of the algorithm in dependence of the number of threads employed}
	\label{fig:scalability}
\end{figure}
