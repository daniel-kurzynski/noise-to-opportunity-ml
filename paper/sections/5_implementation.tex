%!TEX root = ../paper.tex

\section{Implementation}
\label{sec:implementation}

\begin{itemize}
	\item Present data set?
	\item Do classifier optimization
	\item For the concrete data set, how do we split concretely.
\end{itemize}

\subsection{Evaluation data set}
\label{sub:initial_data_set}
We built a prototype of an \nto system.
This prototype is build for the products of a German software company called SAP.
SAP is the biggest software vendor in Europe and builds software for both small and large enterprises.
SAP provided us with nearly 100 brochures about four SAP software products, which are explained in Table~\ref{table:products}.
We chose LinkedIn as the business-oriented social network, as it is the biggest and most popular platform.
We crawled approximately 19,000 LinkedIn posts.
Unfortunately three forth of the brochures are in German language and the LinkedIn posts are written in English.
To solve the language mismatch, we have translated the documents with an automated translator and replaced the German brochures with its translations.
Additionally we have added some book descriptions from Amazon about these products to enlarge the training set.
As already mentioned in the previous sections to extend our training set further, we split the brochures and the Amazon descriptions.
With the help of the splitting the training set grows to 994 document, which we use for the training of the product classifier.

\subsection{Manual annotation of the LinkedIn posts}
\label{sub:manual_annotation_of_the_linkedin_posts}

At the moment, there exists no data set for a demand classification, which can be used for learning and evaluation purposes: we tagged some post manually.
We used an active learning \nr approach to create the gold standard for both training and evaluation:
first, we randomly tagged some posts, then we build a basic classifier, which repeatedly asked for those instances, where it was most unsure about.
Then it reran the classification and asked again.
We have implemented a tagging web app which implements the active learning.
All together, we tagged about 350 LinkedIn posts, both for demand and product.
In an initial step, each post was tagged at least twice to avoid misclassifications because of the opinion of one tagger.
If there were conflicts between the tagging decisions, a third person was tagging again to solve the conflict.
To evaluate the demand classifier, we use a ten-fold cross validation.
Since we have brochures for the products, on which we can learn, we used all annotated posts exclusively for the evaluation in the case of the product classifier.


\subsection{Demand Classifier}

Furthermore, we added the following handcrafted \todo{Better word} features, which have been show to increase the overall performance:
\begin{itemize}
	\item Number of questions in the post. Demand posts usually have more questions.
	\item Number of imperatives in the post. Demand posts usually have more imperatives like ``Please help me out" or ``Give me advice".
	\item Whether an e-mail address was given in the post. Demand posts often give an e-mail address to send answers to.
\end{itemize}

\subsection{Product Classifier}
To be done.

\subsection{Choosing splitting algorithm}
How we split.
