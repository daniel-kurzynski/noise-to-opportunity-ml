%!TEX root = ../paper.tex

\todo{
	\\
	\textbf{Questions} \\
	\begin{itemize}
		\item Ask: Base paper on algorithm or dataset?
	\end{itemize}
	\textbf{Final checks} \\
	\begin{itemize}
		\item "data set" vs "dataset"
	\end{itemize}
}

\todo{Talk about 2-step classifier and show figure one on first page}

\section{Introduction}
\label{sec:introduction}

A large part of today's sales \todo{Better word} happens in online stores.
Many people buy online, as this allows access to a large range of product.
The term E-commerce is used to describe this phenomenon.

As more and more buying decisions happen online, the decision-making process also happens online.
This largely happens in social networks.
Social networks are platforms, where users can post status updates, ask questions, start discussions, or get recommendations.
A special type of social network, so called business-oriented social networking services like LinkedIn\footnote{http://www.linkedin.com/} or Xing\footnote{http://www.xing.com}, especially lend themselves to building business relations and asking for professional advice.
There, users ask for opinions and recommendations on a product.

This offers a huge possibility for companies to identify potential buyers.
Rather than booking advertising time in television or advertising banners on news websites, they can target their salesman on persons, who are more likely to buy a product.
It is a basic truism from marketing, that it is best to target those potential customers, who are already predisposed towards the product. \todo{Reference}

Thus, it makes sense for companies to find these people looking for advice.
However, the large amount of daily status updates every day makes it infeasible to identify this posts manually.
According to LinkedIn, there were 350 users in 2014, each of them posting three posts each week \todo{Random numbers: Check actual values}.
The sheer amount of this shows, that technical assistance is needed.

At the moment, this mostly happens by providing a search engine on the posts.
Salesmen need to manually find good search terms, enter them in the network's search field and then scan the results for potential buyers.
This requires expertise on the hand of the salesman to find good search terms and tweak them accordingly.
Furthermore, scanning through the result lists is cumbersome and tedious.

We propose an approach, which is based on automatic text classification to identify relevant posts for a salesman.
We emply techniques of supervised machine learning to categorize posts.

However, this would require to manually label posts to have a training data set for the learning algorithm.
This is not feasible
We propose an approach, which automatically uses the company's brochures and advertising texts for learning.
\todo{Introduce short corpus problem.}

Section~\ref{sec:background} gives a problem definition and shows related work.
Section~\ref{sec:concept} shows our concept in detail.
Section~\ref{sec:implementation} shows the implementation, and the best configuration for the algorithm.
In Section~\ref{sec:evaluation}, we evaluate our approach.
Section~\ref{sec:conclusion} summarizes and shows our planned future work.

\begin{itemize}
	\item outline your contributions (not your paper$\ddot\smile$)
\end{itemize}
